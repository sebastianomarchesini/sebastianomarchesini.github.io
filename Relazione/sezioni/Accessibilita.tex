\section{Accessibilit\`a}{

	\subsection{Implementazione}{
	Al fine di garantire l'utilizzo del sito ad utenti con disabilità, si sono:
	\begin{itemize}\itemsep0.5pt
		\item Validati i file che compongono il sito con i validatori XHTML1.0 e CSS del W3C;
		\item Separata struttura, presentazione e comportamento;
	%	\item Fonita la possibilità per la tecnologia assistiva di ignorare contenuti non testuali puramente decorativi;
		\item Potuti ridimensionare i testi fino al 200\%, senza perdita di contenuto e funzionalità;
	%	\item La larghezza non supera gli 80 glifi;
		\item Creati contenuti rappresentabili in modalità differenti senza perdere informazioni o la struttura;  %le informazioni, la struttura e la presentazione possono essere determinate programmaticamente o sono disponibili tramite testo; le istruzioni fornite per comprendere ed operare sui contenuti non si basano unicamente su caratteristiche sensoriali dei componenti;
		\item Resi i contenuti in primo piano distinguibili dallo sfondo, grazie a colori diversi. Inoltre si utilizzato un rapporto di contrasto di almeno: 
		\begin{itemize}\itemsep1pt
			\item 7:1 tra testo e sfondo;
			\item 4,5:1 tra testo grande ed immagini contenenti testo grande nella home;
		\end{itemize}
		\item Reso le funzionalità del sito utilizzabili tramite tastiera, mediante i tabindex;
		\item Inseriti:
		\begin{itemize}\itemsep0.5pt
			\item informazioni relative alla posizione dell'utente (breadcrumbs); 
			\item titoli appropriati  per le pagine web; 
			\item testo appropriato e testo alternativo per i collegamenti; 
			\item alternative testuali per il contenuto non testuale:
				\begin{itemize} \itemsep0.5pt
					\item corredata ogni immagine con gli attributi alt e title che la descrivono;
					\item aggiunta di una label ad ogni campo di input della form, in aiuto dello screen reader;
				\end{itemize}
			\item intestazioni ed etichette appropriate; 
			\item l'indicatore del focus nelle interfacce utilizzabili da tastiera; 
			\item sezioni per organizzare il testo.
			\item la lingua predefinita per il contenuto delle pagine; forniti i significati delle abbreviazioni e degli acronimi utilizzati; usati gli attributi \texttt{xml:lang} per definire parole o blocchi in lingua diversa da quella predefinita della pagina;
 			\item Definiti i meta tag: \textit{Description, Keywords, Copyright, Author}
 			\item Un link di ritorno ad inizio pagina.
		\end{itemize} 
		\item Mantenuto un meccanismo di navigazione coerente all'interno delle pagine web del sito;
	%	\item Individuati eventuali errori di inserimento e descritti con del testo; fornite etichette (o istruzioni) per l'input dell'utente.
		%inserito modalità per saltare blocchi di contenuto che si ripetono su più pagine;
	\end{itemize}
	Inoltre, non si sono:
	\begin{itemize}\itemsep1pt
		\item inseriti contenuti audio e video (contenuti multimediali basati sul tempo);
		\item posti vincoli di tempo all'utente per consultare i contenuti o compilare i campi dati;
		\item utilizzato il colore come modalità visiva per rappresentare le informazioni, indicare azioni, elemento di distinzione visiva; inserito contenuto audio eseguito automaticamente all'interno della pagina;
		\item sviluppati contenuti che possano causare attacchi epilettici (non s'è inserito contenuto lampeggiante);
		\item inseriti cambiamenti del contesto su alcun componente che riceve il focus;
	\end{itemize}
		Al fine di facilitare l'utilizzo del sito da parte di utenti con disabilità, si è:
		\begin{itemize}\itemsep1pt

			\item aiutata la navigazione tra le pagine, creando:
			\begin{itemize}\itemsep1pt
				\item \textit{un path o breadcrumb}, per individuare il contesto;
				\item \textit{un menù di link} per mostrare dove si può andare;
				\item \textit{uso dello stesso stile} per tutti i link del sito;
				\item \textit{link di ritorno ad inizio pagina}.
			\end{itemize}
			%\item ridefiniti i tabindex per la navigazione tra i vari link.
			%\item Mantenuti chiari i link
		\end{itemize}
	}
	\subsection{Combinazione dei colori}{
		È stata utlizzato uno schema di colori che garantisse un contrasto di almento 7:1 tra sfondo e testo; per testare le scelte fatte è stato utilizzato il servizio offerto da \url{http://snook.ca/technical/colour_contrast/colour.html}.\\
		Il servizio offerto da \url{http://colorfilter.wickline.org} c'ha permesso di capire come utenti con determinati disturbi visivi visualizzassero il nostro sito.\\
		Di seguito vengono riportati i risultati ottenuti sulla home page.
%		\newpage
		\begin{figure}[H]
%			\centering
			\begin{subfigure}[b]{0.5\textwidth}
				\includegraphics[width=\textwidth]{\docsImg Home.png}
				\vspace{-40pt}
				\caption{Home Page originale}
				\label{Home Page originale}
			\end{subfigure}
			\begin{subfigure}[b]{0.5\textwidth}
				\includegraphics[width=\textwidth]{\docsImg Home.png}
				\vspace{-40pt}
				\caption{Home Page vista da un deutranope}
				\label{Home Page vista da un deutranope}
			\end{subfigure}
			\\
			\\
			\begin{subfigure}[b]{0.5\textwidth}
				\includegraphics[width=\textwidth]{\docsImg Home.png}
				\vspace{-40pt}
				\caption{Home Page vista da un protranope}
				\label{Home Page vista da un protranope}
			\end{subfigure}
			\begin{subfigure}[b]{0.5\textwidth}
			\includegraphics[width=\textwidth]{\docsImg Home.png}
				\vspace{-40pt}
				\caption{Home Page vista da un tritranope}
				\label{Home Page vista da un tritranope}
			\end{subfigure}
			\caption{Homepage vista da persone con problemi nel distinguere i colori}
			\label{fig: Homepage vista da persone con problemi nel distinguere i colori}
		\end{figure}
	}
}