\section{Presentazione}{
	Per presentare al meglio le informazioni disponibili abbiamo posto la nostra attenzione sulla precisione e l'accessibilità.\\
	Avendo separato contenuto, presentazione e struttura, l'uso del codice CSS ha permesso di curare l'aspetto delle pagine; abbiamo usato CSS versione 2, compatibile con la maggior parte dei browser in uso attualmente.\\
	Per rendere migliore la presentazione, abbiamo suddiviso i file CSS in base alle loro funzioni, arrivando ad avere 3 differenti fogli di stile:
	\begin{itemize}
		\item \textbf{home.css}, utilizzato per la maggior parte dei dispositivi con risoluzione maggiore, quali computer portatili e fissi;
		\item \textbf{print.css}, destinato a semplificare la stampa delle pagine.
		Giustifica e modifica il testo, rimpicciolendolo e cambiando il tipo di carattere in uno di più semplice lettura; porta le immagini al centro della pagina. Toglie infine gli sfondi decorativi per ottenere una stampa più chiara.
		
\item \textbf{mobile.css}: viene usato per i dispositivi mobili quali telefoni e tablet che non offrono schermi ampi e richiedono 
		una visualizzazione chiara dell'informazione. Questo foglio di stile viene attivato a risoluzioni inferiori ai 640px.
	\end{itemize}
	Non abbiamo usato font particolari, in questo modo le pagine usano quelli di sistema, garantendo la scalabilità della pagina senza problemi particolari. I caratteri utilizzati hanno una dimensione espresa in "em" al fine di renderli più adattabili alle preferenze dell'utente senza peggiorare l'aspetto del sito.
}